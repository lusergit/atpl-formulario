% Created 2023-07-03 lun 17:01
% Intended LaTeX compiler: pdflatex
\documentclass[11pt]{article}
\usepackage[utf8]{inputenc}
\usepackage[T1]{fontenc}
\usepackage{graphicx}
\usepackage{longtable}
\usepackage{wrapfig}
\usepackage{rotating}
\usepackage[normalem]{ulem}
\usepackage{amsmath}
\usepackage{amssymb}
\usepackage{capt-of}
\usepackage{hyperref}
\usepackage{proof}
\usepackage{mathtools}
\newcommand{\Lang}{\ensuremath{\mathcal{L}}}
\newcommand{\fn}[2]{\ensuremath{\text{fn}\; #1 \; . \; #2}}
\newcommand{\ifc}[3]{\ensuremath{\text{if } #1 \text{ then } #2 \text{ else } #3}}
\newcommand{\app}[2]{\ensuremath{ #1 \; #2}}
\newcommand{\fv}[1]{\ensuremath{\text{fv}\left( #1 \right)}}
\newcommand{\true}{\ensuremath{\mbox{true}}}
\newcommand{\false}{\ensuremath{\mbox{false}}}
\newcommand{\Bool}{\ensuremath{\mbox{Bool}}}
\newcommand{\Nat}{\ensuremath{\mbox{Nat}}}
\newcommand{\unit}{\ensuremath{\mbox{unit}}}
\newcommand{\Unit}{\ensuremath{\mbox{Unit}}}
\newcommand{\inferr}[3]{\ensuremath{\infer[\text{\small{(#1)}}]{#2}{#3}}}
\newcommand{\subs}[2]{\ensuremath{\{ #1 := #2 \}}}
\newcommand{\type}[2]{\ensuremath{ #1 : #2 }}
\newcommand{\Dom}{\ensuremath{\mbox{Dom}}}
\newcommand{\pair}[2]{\ensuremath{( #1 , #2 )}}
\newcommand{\fst}[1]{\ensuremath{ #1 . \__1}}
\newcommand{\snd}[1]{\ensuremath{ #1 . \__2}}
\newcommand{\rec}[1]{\ensuremath{ \{ #1 \} }}
\newcommand{\sel}[2]{\ensuremath{ #1 . #2 }}
\newcommand{\var}[1]{\ensuremath{ \langle #1 \rangle }}
\newcommand{\match}[2]{\ensuremath{ #1 \texttt{ match }\{ #2 \} }}
\newcommand{\case}[3]{\ensuremath{ \texttt{case } #1 = #2 \Rightarrow #3}}
\newcommand{\subt}[2]{\ensuremath{ #1 <: #2 }}
\newcommand{\throw}[1]{\ensuremath{\texttt{throw } #1}}
\newcommand{\tc}[2]{\ensuremath{\texttt{try } #1 \texttt{ catch } #2}}
\let\emptyset\varnothing
\author{Luca Zaninotto -- 2057074}
\date{\today}
\title{ATPL Language and rules}
\hypersetup{
 pdfauthor={Luca Zaninotto -- 2057074},
 pdftitle={ATPL Language and rules},
 pdfkeywords={},
 pdfsubject={},
 pdfcreator={Emacs 28.2 (Org mode 9.5.5)}, 
 pdflang={English}}
\begin{document}

\maketitle
\section*{Language}
\label{sec:org80f7a7a}
\(\Lang\) language is inductively defined by means of the following
grammar:

\begin{center}
\begin{tabular}{rcll}
Terms \(M\) & \(::=\) & \(x\) & variables\\
 & \(\mid\) & \(n \mid\) true \(\mid\) false & integer and boolean constants\\
 & \(\mid\) & \(M + M\) & addition operation\\
 & \(\mid\) & if \(M\) then \(M\) else \(M\) & conditional operation\\
 & \(\mid\) & \(\fn{x}{M}\) & function declaration\\
 & \(\mid\) & \(\app{M}{M}\) & function application\\
\end{tabular}

\end{center}

\subsection*{Free variables}
\label{sec:org8728f7c}
Given a term \(M\), the set of free variables of \(M\) written
\(\fv{M}\)are inductively defined as follows:

\begin{center}
\begin{tabular}{rcl}
\(\fv{x}\) & = & \(\{x\}\)\\
\(\fv{n} = \fv{\false} = \fv{true}\) & = & \(\emptyset\)\\
\(\fv{M + N}\) & = & \(\fv{M} \cup \fv{N}\)\\
\(\fv{\ifc{M_1}{M_2}{M_3}}\) & = & \(\fv{M_1} \cup \fv{M_2} \cup \fv{M_3}\)\\
\(\fv{\fn{x}{M}}\) & = & \(\fv{M} \setminus \{x\}\)\\
\(\fv{\app{M}{N}}\) & = & \(\fv{M}\cup\fv{N}\)\\
\end{tabular}

\end{center}

\section*{Operational Semantics}
\label{sec:org39a16a6}
Defines the smallest binary relation between the terms of \(\Lang\)
that satisfies the following axioms and inference rules
\subsection*{Values}
\label{sec:org9c2b509}
Subset of term corresponding to the possible final states of the
computation, i.e.the results of the program evaluation/execution.
\begin{center}
\begin{tabular}{rcll}
Values \(v\) & \(::=\) & \(n \mid\) true \(\mid\) false & integer and boolean constat\\
 & \(\mid\) & \(\fn{x}{M}\) & function declaration\\
\end{tabular}

\end{center}

\subsection*{Axioms}
\label{sec:orgccb51ad}
\begin{center}
\begin{tabular}{l}
\(\inferr{SUM}{n_1 + n_2 \to n}{}\)\\
\\
\(\inferr{IF-TRUE}{\ifc{\true}{M}{N} \to M}{}\)\\
\\
\(\inferr{IF-FALSE}{\ifc{\false}{M}{N} \to N}{}\)\\
\\
\(\inferr{BETA}{\app{(\fn{x}{M})}{v} \to M\subs{x}{v}}{}\)\\
\end{tabular}

\end{center}

\subsection*{Rules}
\label{sec:orgd0a06e1}
\begin{center}
\begin{tabular}{l}
\(\inferr{IF}{\ifc{M_1}{M_2}{M_3} \to \ifc{M_1'}{M_2}{M_3}}{M_1 \to M_1'}\)\\
\\
\(\inferr{SUM LEFT}{M + N \to M' + N}{M \to M'}\)\\
\\
\(\inferr{SUM RIGHT}{v + M \to v + M'}{M \to M'}\)\\
\\
\(\inferr{APP 1}{\app{M}{N} \to \app{M'}{N}}{M \to M'}\)\\
\\
\(\inferr{APP 2}{\app{v}{M} \to \app{v}{M'}}{M \to M'}\)\\
\end{tabular}

\end{center}

App 1, App 2 e Beta, tutte assieme impostano la convenzione
call-by-value, dove prima di eseguire una funzione su dei
parametri, questi vengono valutati, producendo eventualmente stuck
term \emph{prima} della chiamata a funzione
\section*{Typing}
\label{sec:orgdbd310c}
Types are needed to \emph{classify} values, which in the language are
constants, integers, booleans and functions:
\begin{center}
\begin{tabular}{rcll}
Types \(T\) & \(::=\) & Bool & type of boolean values\\
 & \(\mid\) & Nat & type of integer values\\
 & \(\mid\) & \(T\to T\) & type of functions\\
\end{tabular}

\end{center}
\subsection*{Axioms}
\label{sec:orgdd0e9dd}
\begin{center}
\begin{tabular}{cc}
\(\inferr{T-TRUE}{\Gamma\vdash\type{\true}{\Bool}}{}\) & \(\inferr{T-FALSE}{\Gamma\vdash\type{\true}{\Bool}}{}\)\\
 & \\
\(\inferr{T-INT}{\Gamma\vdash\type{n}{\Nat}}{}\) & \(\inferr{T-VAR}{\Gamma\vdash\type{x}{T}}{}\)\\
\end{tabular}

\end{center}

\subsection*{Rules}
\label{sec:orge75cab7}
\begin{center}
\begin{tabular}{c}
\(\inferr{T-SUM}{\Gamma\vdash\type{(M+N)}{\Nat}}{\Gamma\vdash\type{M}{\Nat} & \Gamma\vdash\type{N}{\Nat}}\)\\
\\
\(\inferr{T-IFTHENELSE}{\Gamma\vdash\type{\ifc{M_1}{M_2}{M_3}}{T}}{\Gamma\vdash\type{M_1}{\Bool} & \Gamma\vdash\type{M_2}{T} & \Gamma\vdash\type{M_3}{T}}\)\\
\\
\(\inferr{T-FUN}{\Gamma\vdash\type{\fn{\type{x}{T_1}}{M}}{T_1 \to T_2}}{\Gamma, \type{x}{T_1}\vdash \type{M}{T_2}}\)\\
\\
\(\inferr{T-APP}{\Gamma\vdash\type{\app{M}{N}}{T_2}}{\Gamma\vdash\type{M}{T_1 \to T_2} & \Gamma\vdash\type{N}{T_1}}\)\\
\end{tabular}

\end{center}

\subsection*{Lemmas}
\label{sec:org8ff9e13}

\subsubsection*{Inversion lemma(s)}
\label{sec:org55ea69b}
\begin{enumerate}
\item If \(\Gamma\vdash\type{\true}{T}\) is derivable, then \(T = \Bool\)
\item If \(\Gamma\vdash\type{\false}{T}\) is derivable, then \(T = \Bool\)
\item If \(\Gamma\vdash\type{n}{T}\) is derivable, then \(T = \Nat\)
\item If \(\Gamma\vdash\type{M+N}{T}\) is derivable, then \(T = \Nat\) and \(\Gamma \vdash \type{M}{\Nat}\),
\(\Gamma\vdash\type{N}{\Nat}\) are derivable.
\item If \(\Gamma\vdash\type{\ifc{M_1}{M_2}{M_3}}{T}\) is derivable,
then \(\Gamma\vdash\type{M_1}{\Bool}\),
\(\Gamma\vdash\type{M_2}{T}\), \(\Gamma\vdash\type{M_3}{T}\)
are derivable judgments
\item If \(\Gamma\vdash\type{x}{T}\) is derivable, then \(\type{x}{T}
       \in \Gamma\)
\item If \(\Gamma\vdash\type{\fn{\type{x}{T_1}}{M}}{T}\) is
derivable, then \(\exists T_2\) s.t. \(T = T_1 \to T_2\) and
\(\Gamma, \type{x}{T_1}\vdash \type{M}{T_2}\) is a derivable
judgment.
\item If \(\Gamma\vdash\type{\app{M}{N}}{T}\) is derivable, then
\(\exists T_1\) s.t. \(\Gamma\vdash\type{M}{T_1\to T_2}\) and
\(\Gamma\vdash\type{N}{T_1}\) are derivable judgments.
\item If \(\Gamma\vdash \type{\unit}{T}\) then \(T = \Unit\)
\item If \(\Gamma\vdash \type{\pair{M_1}{M_2}}{T}\) is derivable,
then \(\exists T_1, T_2 \mid \Gamma\vdash \type{M_1}{T_1}
        \quad \Gamma\vdash \type{M_2}{T_2}\) are derivable.
\item If \(\Gamma\vdash \type{\fst{M}}{T}\) is derivable, then
\(\exists T_1 \mid \Gamma\vdash \type{M}{T * T_1}\) is
derivable.
\item If \(\Gamma\vdash \type{\snd{M}}{T}\) is derivable, then
\(\exists T_1 \mid \Gamma\vdash \type{M}{T_1 * T}\) is
derivable.
\item If \(\Gamma\vdash \type{\rec{\ell_i = M_i \quad^{i\in 1\dots
        n}}} {T}\) is derivable, then \(\forall i \in 1 \dots n \quad
        \exists T_i \mid \Gamma\vdash \type{M_i}{T_i}\) and \(T =
        \rec{\type{\ell_i}{T_i}\quad^{i \in 1\dots n}}\)
\item If \(\Gamma\vdash \type{\sel{M}{\ell_j}}{T}\) is derivable,
then \(\exists n \mid \forall i \in 1\dots n\quad \exists
        \ell_i, T_i \mid \Gamma\vdash \type{M}{\rec{\type{\ell_i}{T_i}
        \quad^{i \in 1\dots n}}}\) and \(T = T_j\).
\end{enumerate}

\begin{itemize}
\item Inversion lemma with subtyping
\label{sec:org315bd31}
We have to add
\begin{enumerate}
\item If \(\Gamma \vdash \type{x}{T}\) then \(\exists x \mid
        \type{x}{S}\) and \(\subt{S}{T}\)
\item If \(\Gamma\vdash\type{\fn{\type{x}{S_1}}{M}}{T}\) is
derivable, then \(\exists T_1, T_2 \mid T = T_1 \to T_2\) with
\(\subt{T_1}{S_1}\) and \(\Gamma,\type{x}{S_1}\vdash
        \type{M}{T_2}\) is derivable
\item If \(\Gamma\vdash\type{\rec{\ell_i = M_i \;^{i\in 1\dots
        n}}}{T}\)is derivable, then \(\exists S_1,\dots,S_n \mid
        \Gamma\vdash \type{M_i}{S_i}\) is derivable for each \(i\in
        1\dots n\) and \(\type{\rec{\type{\ell_i}{S_i}\;^{i\in 1 \dots
        n}}}{T}\)
\item If \(\subt{S}{T_1 \to T_2}\) then \(\exists S_1, S_2\)
s.t. \(S = S_1 \to S_2\) with \(\subt{T_1}{S_1},
        \subt{S_2}{T_2}\)
\item If \(\subt{S_1\to S_2}{T}\) then \(\exists T_1, T_2\) s.t. \(T
        = T_1 \to T_2\) with \(\subt{T_1}{S_1}, \subt{S_2}{T_2}\)
\item If \(\subt{S}{\rec{\type{\ell_i}{T_i}\;^{i\in I}}}\) then
\(S=\rec{\type{\ell_i}{S_i}\;^{i\in I \cup I'}}\) with
\(\subt{S_i}{T_i}\) for all \(i \in I\)
\item If \(\subt{\rec{\type{\ell_i}{S_i}\;^{i\in I}}}{T}\) then
\(T=\rec{\type{\ell_i}{T_i}\;^{i\in I'}}\) with \(I\subseteq
        I'\) and \(\subt{S_i}{T_i}\) for all \(i \in I\)
\end{enumerate}
\end{itemize}

\subsubsection*{Canonical form lemma(s)}
\label{sec:org4b40b87}
\begin{enumerate}
\item if \(v\) is a value and \(\emptyset \vdash \type{v}{\Bool}\)
then \(v\) is either \texttt{true} or \texttt{false}.
\item if \(v\) is a value and \(\emptyset \vdash \type{v}{\Nat}\)
then \(v\) is an integer constant \(n\).
\item if \(v\) is a value and \(\emptyset \vdash \type{v}{T_1 \to
       T_2}\) then \(v\) is \(\fn{\type{x}{T_1}}{M}\).
\item if \(v\) is a value and \(\emptyset \vdash \type{v}{\Unit}\) then
\(v\) is the \(\unit\) value.
\item if \(v\) is a value and \(\emptyset \vdash \type{v}{T_1 * T_2}\)
then \(v\) is in the shape \(\pair{v_1}{v_2}\).
\item if \(v\) is a value and \(\emptyset \vdash
       \type{v}{\rec{\type{\ell_i}{T_i}\quad^{i\in 1 \dots n}}}\) then
\(v\) is in the shape \(\rec{\ell_i = v_i\quad^{i\in 1 \dots
       n}}\).
\item if \(v\) is a value and \(\emptyset \vdash
       \type{v}{\var{\type{\ell_i}{T_i}\quad^{i\in 1 \dots n}}}\) then
\(v\) is in the shape \(\var{\ell_i = v_i}\) with \(i\in 1\dots
       n\).
\end{enumerate}

\subsubsection*{Permutation lemma}
\label{sec:org63792d8}
If \(\Gamma\vdash\type{M}{T}\) is a derivable judgment, and
\(\Delta\) is a permutation of \(\Gamma\), then
\(\Delta\vdash\type{M}{T}\) is derivable with proof tree of the
same height.

The proof works by induction on the height of \(\Gamma\vdash
    \type{M}{T}\).

\subsubsection*{Weakening lemma}
\label{sec:org4aeecbd}
If \(\Gamma\vdash\type{M}{T}\) is a derivable judgment and \(x
    \not\in \Dom (\Gamma)\), then
\(\Gamma,\type{x}{S}\vdash\type{M}{T}\)

The proof works by induction on the height of \(\Gamma\vdash
    \type{M}{T}\).

\subsubsection*{Substitution lemma}
\label{sec:orgdf4cee5}
If \(\Gamma,\type{x}{S}\vdash\type{M}{T}\) and
\(\Gamma\vdash\type{N}{S}\), then \(\Gamma \vdash \type{M
    \subs{x}{N}}{T}\)

The proof works by induction on the height of \(\Gamma,
    \type{x}{S}\vdash \type{M}{T}\).

\subsection*{Theorems}
\label{sec:org928ed05}

\subsubsection*{Progress theorem}
\label{sec:orgb234843}
Let \(M\) be a closed and well-typed term, i.e. \(\emptyset
    \vdash\type{M}{T}\), then either \(M\) is a value or \(\exists M'\)
s.t. \(M\to M'\).

The proof works by induction on the height of \(\Gamma\vdash
    \type{M}{T}\).

\subsubsection*{Subject Reduction or Type Preservation}
\label{sec:org82b4c71}
If \(\Gamma\vdash\type{M}{T}\) and \(M\to M'\), then \(\Gamma
    \vdash \type{M'}{T}\)

The proof works by induction on the height of \(M\to M'\).

\begin{itemize}
\item \textbf{Corollary}:
\label{sec:org335b7ab}
If \(\emptyset\vdash\type{M}{T}\) and \(M \to^* M'\), then
\(\emptyset\vdash\type{M'}{T}\)
\end{itemize}

\subsubsection*{Safety}
\label{sec:org7c73e2e}
If \(M\) is a closed and well-typed term, then it does not reduce
to a stuck term. That is, let \(M\) be s.t. \(\emptyset \vdash
    \type{M}{T}\) and \(M\to* M'\) s.t. \(M'\not\to\), then \(M'\) is
a value

The proof just works because of the latter theorems.

\section*{Extensions}
\label{sec:org2127e27}
\subsection*{Unit type}
\label{sec:org202fb69}
\begin{center}
\begin{tabular}{llll}
M := \(\dots \mid\) unit & \(v := \dots \mid\) unit & \(T := \dots \mid\) Unit & \(\inferr{T-UNIT}{\Gamma \vdash \type{\unit}{\Unit}}{}\)\\
\end{tabular}

\end{center}

Thanks to the unit type, we can introduce syntactic sugar
\(M_1;M_2\), which expands to
\(\app{(\fn{\type{x}{\Unit}}{M_2})}{M_1}\) with \(x\) fresh,
i.e. \(x\not\in fv(M_2)\)

We can also optionally introduce the following derivation rules to
concatenate terms with the new construct:

\begin{center}
\begin{tabular}{ll}
\(\inferr{SEQ}{M_1;M_2 \to M_1';M_2}{M_1 \to M_1'}\) & \(\inferr{SEQ-NEXT}{\unit;M \to M}{}\)\\
\end{tabular}

\end{center}

\[\inferr{T-SEQ}{\Gamma \vdash \type{M_1;M_2}{T}}{\Gamma \vdash
   \type{M_1}{\Unit} & \Gamma \vdash \type{M_2}{T}}\]

\subsection*{Pairs}
\label{sec:orgbc591ad}
\begin{center}
\begin{tabular}{lcr}
\(M\) & := & \(\dots \mid \pair{M_1}{M_2} \mid \fst{M_1} \mid \snd{M_1}\)\\
\end{tabular}

\end{center}

Pairs have a new \emph{product type}: \(T_1 * T_2\), and pairs values
are in the shape \(v_1, v_2\).

\subsubsection*{Axioms}
\label{sec:org5c64e1b}

\begin{center}
\begin{tabular}{ll}
\(\inferr{PAIR-1}{\fst{\pair{v_1}{v_2}} \to v_1}{}\) & \(\inferr{PAIR-2}{\snd{\pair{v_1}{v_2}} \to v_2}{}\)\\
\end{tabular}

\end{center}

\subsubsection*{Rules}
\label{sec:orgd9ba473}

\begin{center}
\begin{tabular}{ll}
\(\inferr{PROJECT-1}{\fst{M} \to \fst{M'}}{M \to M'}\) & \(\inferr{PROJECT-2}{\snd{M} \to \snd{M'}}{M \to M'}\)\\
 & \\
\(\inferr{EVAL-PAIR-1}{\pair{M_1}{M_2} \to \pair{M_1'}{M_2}}{M_1 \to M_1'}\) & \(\inferr{EVAL-PAIR-2}{\pair{v}{M_2} \to \pair{v}{M_2'}}{M_2 \to M_2'}\)\\
\end{tabular}

\end{center}

\subsubsection*{Typing rules}
\label{sec:orgeab0869}

\[\inferr{T-PAIR}{\Gamma\vdash\type{\pair{M_1}{M_2}}{T_1 *
    T_2}}{\Gamma\vdash\type{M_1}{T_1} & \Gamma\vdash \type{M_2}{T_2}}\]

\begin{center}
\begin{tabular}{ll}
\(\inferr{T-PROJECT-1}{\Gamma \vdash \type{\fst{M}}{T_1}}{\Gamma\vdash\type{M}{T_1 * T_2}}\) & \(\inferr{T-PROJECT-2}{\Gamma \vdash \type{\snd{M}}{T_2}}{\Gamma\vdash\type{M}{T_1 * T_2}}\)\\
\end{tabular}

\end{center}

\subsection*{Records}
\label{sec:org1e79f67}

\begin{center}
\begin{tabular}{lcl}
\(M\) & ::= & \(\dots \mid \rec{\ell_i = M_i ^{i \in 1..n}}\)\\
 &  & \\
\(v\) & ::= & \(\dots \mid \rec{\ell_i = v_i ^{i \in 1..n}}\)\\
 &  & \\
\(T\) & ::= & \(\dots \mid \rec{\ell_i : T_i ^{i \in 1..n}}\)\\
\end{tabular}

\end{center}

\subsubsection*{Axioms}
\label{sec:org724ab5d}
\[\inferr{SELECT}{\sel{\rec{\ell_i = v_i \quad^{i \in
    1..n}}}{\ell_j} \to v_j}{j\in 1..n}\]

\subsubsection*{Rules}
\label{sec:org3432229}
\begin{center}
\begin{tabular}{l}
\(\infer[\text{(EVAL-SELECT) }]{\sel{M}{\ell} \to \sel{M'}{\ell}} {M\to M'}\)\\
\\
(EVAL RECORD)\\
\(\infer{\rec{\ell_i = v_i ^{i\in 1\dots j-1}, \ell_j = M_j, \ell_k = M_k^{k \in j+1 \dots n}} \to \rec{\ell_i = v_i ^{i\in 1\dots j-1}, \ell_j = M_j', \ell_k = M_k^{k \in j+1 \dots n}}}{M_j \to M_j'}\)\\
\end{tabular}

\end{center}

\subsubsection*{Typing rules}
\label{sec:orgc5c1e22}
\begin{center}
\begin{tabular}{ll}
(T-RECORD) & (T-SELECT)\\
 & \\
\(\infer{\Gamma\vdash\type{\rec{\ell_i = M_i\quad ^{i\in 1 .. n}}}{\rec{\type{\ell_i}{T_i}\quad^{i\in 1..n}}}}{\forall i \in 1 .. n \quad \Gamma \vdash M_i : T_i}\) & \(\infer{\Gamma\vdash \type{M.\ell_j}{T_j}}{\Gamma\vdash \type{M}{\rec{\type{\ell_i}{T_i}\quad^{i \in 1..n}}}}\)\\
\end{tabular}

\end{center}

with \(j \in \{1, \dots, n\}\)

\subsection*{Variant types}
\label{sec:orgacb4f53}
\begin{center}
\begin{tabular}{lcl}
\(M\) & ::= & \(\dots \mid \var{\ell = M}\)\\
 &  & \\
\(v\) & ::= & \(\dots \mid \var{\ell = v}\)\\
 &  & \\
\(T\) & ::= & \(\dots \mid \var{\type{\ell_i}{T_i}^{\quad i\in 1\dots n}}\)\\
\end{tabular}

\end{center}

\subsubsection*{Axioms}
\label{sec:orgf6c0af2}
\[\infer{\match{\var{\ell_j =
    v_j}}{\case{\ell_i}{x_i}{M_i}^{\quad i \in 1\dots n}}\to
    M_j\subs{x_j}{v_j}}{\text{(MATCH)}}\]
\subsubsection*{Rules}
\label{sec:org3de8fb5}
\begin{itemize}
\item (RED-MATCH) \[\infer{\match{M}{\case{\ell_i}{x_i}{M_i}^{\quad i
      \in 1\dots n}}\to \match{M'}{\case{\ell_i}{x_i}{M_i}^{\quad i
      \in 1\dots n}} }{M\to M'}\]
\item (VARIANT) \[\infer{\var{\ell = M} \to \var{\ell = M'}}{M\to
      M'}\]
\end{itemize}
\subsubsection*{Typing rules}
\label{sec:org9a9f59a}
\begin{itemize}
\item (TYPE-VARIANT) \[\infer[j\in 1 \dots n]{\Gamma \vdash
      \type{\var{\ell_j = M}}{\var{\type{\ell_i}{T_i}^{\quad i \in 1
      \dots n}}}}{\Gamma \vdash \type{M}{T_j}}\]
\item (TYPE-MATCH)
\[\infer{\Gamma\vdash\match{M}{\case{\ell_i}{x_i}{M_i}^{\quad i
      \in 1\dots n}}}{\Gamma\vdash
      \type{M}{\var{\type{\ell_i}{T_i}^{i\in 1\dots n}}} & \forall i
      \in 1 \dots n\quad \Gamma, \type{x_i}{T_i}\vdash
      \type{M_i}{T}}\]
\end{itemize}

\subsection*{Subtyping}
\label{sec:org94436cc}
No terms, values or types are added, justo new rules to embed the
concept of \emph{subtyping}
\subsubsection*{Axioms}
\label{sec:org9436b20}
(REFLEX) \[\infer{\subt{T}{T}}{}\]
\subsubsection*{Rules}
\label{sec:org32d7f76}
The main rule added is (SUBSUMPTION), it allows for generic
subtyping. Attention! with this rule we loss the unicity of
typing!

\begin{center}
\begin{tabular}{ll}
\(\inferr{SUBSUMPTION}{\Gamma\vdash \type{M}{T}}{\Gamma \vdash \type{M}{T} & \subt{S}{T}}\) & \(\inferr{TRANS}{\subt{S}{T}}{\subt{S}{U} & \subt{U}{T}}\)\\
\end{tabular}

\end{center}

\subsubsection*{Record-sepcific rules}
\label{sec:org4152d5e}
We're introducing record-specific rules, since is the data
structures which allows us to encode sort of \emph{objects} into our
language

\begin{itemize}
\item (SUB-WIDTH) \[\infer{\subt{\rec{\type{\ell_i}{T_i}^{\quad i \in
      I\cup I'}}}{\rec{\type{\ell_i}{T_i}^{\quad i \in I}}}}{}\]
\item (SUB-DEPTH) \[\infer{ \subt{ \rec{ \type{\ell_i}{S_i}^{\quad i
      \in I}}}{\rec{\type{\ell_i}{T_i}^{\quad i \in I}}}}
      {\subt{S_i}{T_i}\quad \forall i \in I}\]
\end{itemize}
\subsection*{Exceptions}
\label{sec:org07a8be7}
\begin{center}
\begin{tabular}{lcl}
\(M\) & ::= & \(\dots \mid \throw{M} \mid \tc{M_1}{M_2}\)\\
\end{tabular}

\end{center}

\subsubsection*{Axioms}
\label{sec:org95db2da}
\begin{itemize}
\item (TRY-VAL) \[\infer{\tc{v}{M}\to v}{}\]
\item (RAISE-2) \[\infer{\throw{\throw{v}} \to v}{}\]
\item (RAISE APP 1) \[\infer{\app{\throw{v}}{M} \to \throw{v}}{}\]
\item (RAISE APP 2) \[\infer{\app{v_1}{\throw{v_2}} \to
      \throw{v_2}}{}\]
\item (RAISE IF-THEN-ELSE) \[\infer{\ifc{\throw{v}}{M}{N}\to
      \throw{v}}{}\]
\item (RAISE SUM 1) \[\infer{\throw{v} + M \to \throw{v}}{}\]
\item (RAISE SUM 2) \[\infer{v_1 + \throw{v_2} \to \throw{v_2}}{}\]
\end{itemize}
\subsubsection*{Rules}
\label{sec:org4bc0b88}
\begin{itemize}
\item (TRY) \[\infer{\tc{M}{N} \to \tc{M'}{N}}{M\to M'}\]
\item (RAISE 1) \[\infer{\throw{M} \to \throw{M'}}{M\to M'}\]
\end{itemize}
\subsubsection*{Typing rules}
\label{sec:org568c11d}
\begin{itemize}
\item (T RAISE) \[\infer{\Gamma\vdash \type{\throw{M}}{T}}
      {\Gamma\vdash \type{M}{T_{exn}}}\]
\item (T RAISE)
\[\infer{\Gamma\vdash\type{\tc{M}{N}}{T}}{\Gamma\vdash\type{M}{T}
      & \Gamma\vdash \type{N}{(T_{exn}\to T)}}\]
\end{itemize}
\subsection*{Featherweight Java}
\label{sec:orgb3c980b}
\begin{center}
\begin{tabular}{rcl}
Class declaration CL & ::= & class C extends D \{ A\textasciitilde{} f; K M\textasciitilde{} \}\\
 &  & \\
Constructor Declaration K & ::= & C(A\textasciitilde{} g, B\textasciitilde{} f) \{ super(\(g~\)); this.f\textasciitilde{} = f\textasciitilde{} \}\\
 &  & \\
Method Declaration M & ::= & C m(A\textasciitilde{} x) \{ return t; \}\\
\end{tabular}

\end{center}

\begin{center}
\begin{tabular}{rcllr}
Terms \(t\) & ::= & x & variables & \\
 &  & \(t\).f & field selection & \\
 &  & \(t\).m(t*) & method invocation & Values \(u,v ::=\) new C(\(v\))\\
 &  & new C(t*) & object creation & \\
 &  & \((C)t\) & cast & \\
\end{tabular}

\end{center}
\end{document}
